\documentclass[dvips,12pt]{article}



\usepackage[pdftex]{graphicx}
\usepackage{url}



\setlength{\oddsidemargin}{0.25in}
\setlength{\textwidth}{6.5in}
\setlength{\topmargin}{0in}
\setlength{\textheight}{8.5in}


\title{\bf Team ID:1054}
\author{}
\date{}

\newcommand{\namelistlabel}[1]{\mbox{#1}\hfil}
\newenvironment{namelist}[1]{%1
\begin{list}{}
    {
        \let\makelabel\namelistlabel
        \settowidth{\labelwidth}{#1}
        \setlength{\leftmargin}{1.1\labelwidth}
    }
  }{
\end{list}}

\begin{document}
\maketitle

\begin{namelist}{xxxxxxxxxxxx}
\item[{\bf Title:}]
	AQUA-Bot: A Crop Watering Mobile Robot For precision
Irrigation.
\item[{\bf Members:}]
	Nishant Mishra
	\item Piyush Bansal
	\item Mayank Bansal
	\item Kashish Awasthi
\item[{\bf Mentor:}]
	Mrs.Bindu Rani
\end{namelist}


\section*{Project Keywords} eyic2019-20,More productivity,Sustainable irrigation,IOT,Saving water,Good crops,SDG


\section*{Project Introduction} We are running short on water in many parts of the world. Droughts are becoming more common and
severe, and the costs of water and food are going up. Farmers and gardeners urgently need to produce more
with less water. India uses about 75 percent of its fresh water for irrigation, still farmers are facing the problem of
reduced quality of crops due to inadequate irrigation, and we are thriving for fresh drinking water. Thus,
finding a unique way is need of the hour so that we can get “More Crop, Per Drop”. The objective is to make
such a robot which waters the crops as per their requirements. Here by requirement we mean that plant will
get that much water, that they need for their good development. Here our Robot has sensors like soil
moisture sensor, temperature sensor, and a DHT22 sensor which are used to sense the moisture of the soil,
temperature of the soil and the humidity of the surroundings. These conditions are analysed by our robot,
using the algorithm we have uploaded in the microcontroller, If the plant requires water our robot will
irrigate the plant by pouring water from the inbuilt water tank on the robot.
\section*{Project Literature Survey}
Micro irrigation has seen a steady growth over the years. Since 2005, area covered under micro irrigation
systems has grown at a CAGR of 9.6 percent and reached 7.73 million hectares. However, the potential area which
can be covered under micro – irrigation was totaled to 69.5 million hectares in 2015, As now a days farmers
are facing problems in proper irrigation of their crops, they are using some micro irrigation systems like drip
irrigation or Agricultural robots are gaining traction among the farmers, owing to the need for producing
sprinkler systems which are no matter good systems but not portable. Thus, there is a good scope for such
watering robots who will give water to the crops as per their requirement, and controlled by phones by the
farmers like a RC toy.

\section*{Hardware Requirements}
\begin{itemize}
 \item 1. NodeMCU ESP8266.
 \item 2. NodeMCU ESP8266 Serial Port Baseboard Lua WIFI Development Board.
\item 3. Temperature and Humidity Sensor DHT22.
\item 4. Soil Moisture Sensor.
\item 5. Waterproof Digital Temperature Sensor Probe DS18B20.
\item 6. DC 12V 2CH 2 Channel Isolated Optocoupler High/Low Level Trigger Relay Module.
\item 7. Mini DC Water Pump.
\item 8. 1602 (16x2) LCD Display with I2C/IIC interface.
\item 9. Jumper Wires.
\item 10. LED’s.
\item 11. Buttons.
\item 12.Resistance box.
\end{itemize}
\section*{Software Requirements}
\begin{itemize}
\item 1. Arduino IDE.
\item 2. Fritzing.
\item 3. Circuito.io.
\item 4. Blynk(Android App.).
\end{itemize}
\section*{Implementations}
The farmers will be given an IOT controlled robot (AQUA-Bot) which can be controlled with the
smartphones like a remote-controlled car toy. This robot will be equipped with a smart irrigation module,
which consists of sensors like soil moisture sensor, soil Temperature sensors and a DHT22 sensor. All these
sensors will be connected to a NodeMCU ESP8266 wifi microcontroller and a water pump will also be
connected to this microcontroller.
All the sensors in the smart irrigation module will take the data from the soil and the plant’s surroundings,
and this data will be analysed by the algorithm we have uploaded in the microcontroller and will be sent to
the Blynk Cloud. this data will be sent to the android application (Blynk) and can be analysed by the farmers
too, now for example if the condition is true (i.e. Moisture of soil is less than 50percent) then a notification will be
sent to the farmer that moisture is low, please turn on the water pump, now farmer can decide whether he
wants to water the plant or not. Let’s take an, another condition (i.e. Moisture of soil is more than 50% but
the humidity of surroundings is low and temperature is high) then farmer will analyse the condition and can
still water the plant even if the soil moisture is sufficient.
The image given below is the block diagram of the smart irrigation module.

\begin{center}
\resizebox{7in}{!}{\includegraphics*{fk.jpg}}
\end{center}


\bf Flowchart:
\begin{center}
\resizebox{7in}{!}{\includegraphics*{fk2.jpg}}
\end{center}

\section*{Feasibility}
The current micro-irrigation technologies which are being used by the farmers are drip irrigation, sprinkler
systems etc. These techniques are no doubt are good but there, limitation is they are not portable and they do
not analyse the water requirements of plants according to the environmental conditions. In order to introduce
a new, unique and portable way to irrigate the field using less water, we have come up with our idea of
“Crop watering mobile robot for precision irrigation” whom we have named AQUA-Bot. Our idea can be
installed in the Agricultural robots present in the market making them AGRIBOTS 2.0, as, in agribots now a
days there are soil sensors, AI based weeding mechanisms etc. but there is no any way to water the plants
using the agribot. Thus, our idea will definitely become a milestone in irrigation sector and is market
feasible.

 
\begin{thebibliography}{99}

\bibitem ihttps://www.electronicwings.com/nodemcu/soil-moisture-sensor-interfacing-with-nodemcu
\bibitem ihttps://www.14core.com/wiring-the-esp8266-12e-remote-soil-moisture-temperature-humidity-monitor/
\bibitem ihttps://www.teachmemicro.com/nodemcu-esp8266-dht22-interfacing/
\bibitem ihttps://www.instructables.com/id/NodeMCU-ESP8266-WiFi-Robot-Car-Controlled-by-Appli/
\bibitem ihttps://securingwaterforfood.org/innovators
\bibitem ihttps://blynk.io/
\bibitem ihttps://create.arduino.cc/projecthub
\end{thebibliography}



\end{document}